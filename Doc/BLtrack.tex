\documentclass[a4paper]{report}
\title{BLtrack documentation}
\author{Kyrre Sjobak}

\usepackage{hyperref}
\usepackage{algorithmic}
%\usepackage{todonotes}
\usepackage{amsmath}

\begin{document}
\maketitle
\newpage
\tableofcontents

\chapter{Introduction}

The BLtrack program is written to study the effect of beam loading in crab cavities.
It can track several bunches of particles in a 6D phase space, and allows the elements to change dynamically over time, including responding dynamically to the beam.
Most of the BLtrack input is meant to be exported from MadX, permitting easy set-up of a new configuration.

\section{Program structure}
The program structure is split in two main classes: \texttt{Beam} and \texttt{Ring}.
These then hold objects of classes inheriting from the \texttt{Bunch} and \texttt{Element} classes.

To start tracking, the \texttt{Machine} object has a method \texttt{track(beam,turns)} which tracks the provided \texttt{Beam} object through the ring the given number of turns.
The algorithm used to track all the particles through all the elements for the required number of turns is very simple:
\begin{algorithmic}
  \FOR{$t=0$ \TO $\mathtt{turns}-1$}
    \FOR{\texttt{element} in \texttt{elements}}
      \FOR{\texttt{bunch} in \texttt{bunches}}
        \STATE \texttt{bunch}.\texttt{particles} = \texttt{element}.\texttt{track}(\texttt{bunch},$t$)
      \ENDFOR
    \ENDFOR
  \ENDFOR
\end{algorithmic}

Finally there is the main executable, \texttt{BLtrack.py}.
This reads an input file, parses some sections of it directly, and passes the \texttt{BEAM} and \texttt{RING} blocks on to the relevant constructors.

\section{Input file structure}
The input file is based on newline-separated statements, identified by the first word in that statement.
The keywords are:
\begin{description}
\item[/\textit{any text}] A comment, which is ignored.
\item[TURNS \textit{int}] The number of turns to track.
\item[BEAM] Indicates the start of a \texttt{BEAM} block, which is parsed as described in Section~\ref{sec:beam}.
\item[RING] Indicates the start of a \texttt{RING} block, which is parsed as described in Section~\ref{sec:machine}.
\item[IMPORTFILE \textit{filename}] Accepted inside a \texttt{BEAM} or \texttt{RING} block, causing the content inside the file (one or more lines) to be read and used in the place of this line.
\item[NEXT] Indicates the end of the current block, and transfers the content of the block (including the content of any \texttt{IMPORTFILE}s) to the constructor of the \texttt{Beam} or \texttt{Ring} for parsing.
\end{description}

\chapter{Beam definition}
\label{sec:beam}

%\todo[inline]{Parsing}

Each tracked particle is described by its position in a 6D phase-space.
The coordinates and their units are the same as in MadX (CITE MadX manual):
\begin{description}
\item[X]  Position in the horizontal direction [m].
\item[PX] Fractional transverse momentum in the horizontal direction $p_x/p_0$.
\item[Y]  Position in the horizontal direction [m].
\item[PY] Fractional transverse momentum in the vertical direction $p_y/p_0$.
\item[T]  Position relative to a reference particle, i.e.\ $\mathrm{T}=-c t$ [m]. Thus a positive T implies that the particle arrives ahead of the reference particle ($t=0$).
\item[PT] Normalized energy error $\Delta E/(p_s c)$.
\end{description}
Here $p_0$ is the design momentum of the machine, and $p_s$ the reference momentum.
They are related through the variable $\mathrm{DELTAP} = (p_s-p_0)/p_0$, which is assumed to be 0 $\Rightarrow p_s=p_0$.

The \texttt{Beam} oject holds one or more \texttt{Bunch} objects, which again holds the actual particles to be tracked. Further, it reads the beams total (design) energy $\mathtt{E0}$ [eV] from the input file, and uses it to compute the following related variables:
\begin{equation}
  \mathtt{gamma0} = \gamma_0 = E_0/m_0
\end{equation}
\begin{equation}
  \mathtt{beta0} = \beta_0 = \sqrt{\left(1-1/\gamma_0\right) \left(1+1/\gamma_0\right)}
\end{equation}
\begin{equation}
  \mathtt{p0} = p_0 = \sqrt{\left(E_0-m_0\right) \left(E_0-m_0\right)}
  \label{eq:designmomentum}
\end{equation}
Here the variable m0 [eV/c\textsuperscript{2}] is the beam particle's rest mass, which by default is equal to the proton mass at 938.272 MeV/c\textsuperscript{2}.

TODO: Multiple particles / storage of particle array, coupling to SectorMapMatrix (matrix orientation) \ldots

TODO: Input file reading.

\chapter{Machine description}
\label{sec:machine}

%\todo[inline]{Parsing}


\section{Elements}
All elements inherit from the same \texttt{Element} class, and must inplement a constructor and the methods \texttt{track}(\texttt{bunch}, \texttt{turn}), \texttt{getMatrix}(), and \texttt{\_\_str\_\_}().

\subsection{SectorMapMatrix}
This element describes a part of the machine as a $6 \times 6$ matrix, which when multiplied with the particle vector describe the linear effect of a part of the lattice.
The matrix can be extracted from MadX as a one turn map or as a sectormap (see section~\ref{sec:MadX})

TODO: Input

\subsection{RFcavitiy}

This element implements an accelerating RF cavity.
The necessary input is the voltage $V$ [Volts], wavelength $\lambda$ [m], and phase $\phi$ [radians].
The applied kick is then
\begin{equation}
  \Delta \left(\frac{\Delta E}{p_s c}\right) = \frac{\sin\left(\frac{2\pi \mathtt{T} }{\lambda}+\phi\right)}{p_0 c}~,
\end{equation}
where it is assumed that $p_s = p_0$.
The design momentum is taken from the beam object, where it is calculated as described in Equation~\eqref{eq:designmomentum}.


TODO: Input

TODO: Matrix implementation

\subsection{CrabCavity}
\subsection{PrintMean}
\subsection{PrintBunch}

\appendix
\chapter{Creating a machine definition from MadX}
\label{sec:MadX}

\chapter{Testing for correctness}
\section{Betatron tune}

\section{Synchrotron tune}

\end{document}