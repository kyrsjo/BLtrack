\documentclass[a4paper]{report}
\title{BLtrack documentation}
\author{Kyrre Sjobak}

\usepackage{hyperref}
\usepackage{algorithmic}
%\usepackage{todonotes}
\usepackage{amsmath}
\usepackage{amsfonts}

\begin{document}
\maketitle
\newpage
\tableofcontents

\chapter{Introduction}

The BLtrack program is written to study the effect of beam loading in crab cavities.
It can track several bunches of particles in a 6D phase space, and allows the elements to change dynamically over time, including responding dynamically to the beam.
Most of the BLtrack input is meant to be exported from MadX, permitting easy set-up of a new configuration.

\section{Program structure}
The program structure is split in two main classes: \texttt{Beam} and \texttt{Ring}.
These then hold objects of classes inheriting from the \texttt{Bunch} and \texttt{Element} classes.

To start tracking, the \texttt{Machine} object has a method \texttt{track(beam,turns)} which tracks the provided \texttt{Beam} object through the ring the given number of turns.
The algorithm used to track all the particles through all the elements for the required number of turns is very simple:
\begin{algorithmic}
  \FOR{$t=0$ \TO $\mathtt{turns}-1$}
    \FOR{\texttt{element} in \texttt{elements}}
      \FOR{\texttt{bunch} in \texttt{bunches}}
        \STATE \texttt{bunch}.\texttt{particles} = \texttt{element}.\texttt{track}(\texttt{bunch},$t$)
      \ENDFOR
    \ENDFOR
  \ENDFOR
\end{algorithmic}

Finally there is the main executable, \texttt{BLtrack.py}.
This reads an input file, parses some sections of it directly, and passes the \texttt{BEAM} and \texttt{RING} blocks on to the relevant constructors.

\section{Input file structure}
The input file is based on newline-separated statements, identified by the first word in that statement.
The keywords are:
\begin{description}
\item[/\textit{any text}] A comment, which is ignored.
\item[TURNS \textit{int}] The number of turns to track.
\item[BEAM] Indicates the start of a \texttt{BEAM} block, which is parsed as described in Section~\ref{sec:beam}.
\item[RING] Indicates the start of a \texttt{RING} block, which is parsed as described in Section~\ref{sec:machine}.
\item[IMPORTFILE \textit{filename}] Accepted inside a \texttt{BEAM} or \texttt{RING} block, causing the content inside the file (one or more lines) to be read and used in the place of this line.
\item[NEXT] Indicates the end of the current block, and transfers the content of the block (including the content of any \texttt{IMPORTFILE}s) to the constructor of the \texttt{Beam} or \texttt{Ring} for parsing.
\end{description}

\chapter{Beam definition}
\label{sec:beam}

%\todo[inline]{Parsing}

Each tracked particle is described by its position in a 6D phase-space.
The coordinates and their units are the same as in MadX (CITE MadX manual):
\begin{description}
\item[X]  Position in the horizontal direction [m].
\item[PX] Fractional transverse momentum in the horizontal direction $p_x/p_0$.
\item[Y]  Position in the horizontal direction [m].
\item[PY] Fractional transverse momentum in the vertical direction $p_y/p_0$.
\item[T]  Position relative to a reference particle, i.e.\ $\mathtt{T}=-c t$ [m]. Thus a positive \texttt{T} implies that the particle arrives ahead of the reference particle ($t=0$).
\item[PT] Normalized energy error $\Delta E/(p_s c)$.
\end{description}
Here $p_0$ is the design momentum of the machine, and $p_s$ the reference momentum.
They are related through the variable $\mathtt{DELTAP} = (p_s-p_0)/p_0$, which is assumed to be 0, which implies that $p_s = p_0$.

The \texttt{Beam} oject holds one or more \texttt{Bunch} objects, which again holds the actual particles to be tracked. Further, it reads the beams total (design) energy $\mathtt{E0}$ [eV] from the input file, and uses it to compute the following related variables:
\begin{equation}
  \mathtt{gamma0} = \gamma_0 = E_0/m_0
\end{equation}
\begin{equation}
  \mathtt{beta0} = \beta_0 = \sqrt{\left(1-1/\gamma_0\right) \left(1+1/\gamma_0\right)}
\end{equation}
\begin{equation}
  \mathtt{p0} = p_0 = \sqrt{\left(E_0-m_0\right) \left(E_0-m_0\right)}
  \label{eq:designmomentum}
\end{equation}
Here the variable m0 [eV/c\textsuperscript{2}] is the beam particle's rest mass, which by default is equal to the proton mass at 938.272 MeV/c\textsuperscript{2}.

TODO: Multiple particles / storage of particle array, coupling to SectorMapMatrix (matrix orientation) \ldots

TODO: Input file reading.

\chapter{Machine description}
\label{sec:machine}

%\todo[inline]{Parsing}


\section{Elements}
All elements inherit from the same \texttt{Element} class, and must inplement a constructor and the methods \texttt{track}(\texttt{bunch}, \texttt{turn}), \texttt{getMatrix}(), and \texttt{\_\_str\_\_}().

\subsection{SectorMapMatrix}
This element describes a part of the machine as a $6 \times 6$ matrix $\mathbf{R}$, which when multiplied with the particle vector describe the linear effect of a part of the lattice.
The matrix can be extracted from MadX as a one turn map or as a sectormap (see section~\ref{sec:MadX})

TODO: Input

\subsection{RFcavity}
\label{sec:elements:RFcavity}

This element implements an accelerating RF cavity.
The necessary input is the voltage $V_0$ [Volts], wavelength $\lambda$ [m], and phase $\phi$ [radians], and the voltage seen by the particle is
\begin{equation}
  V(t) = V_0 \sin\left( -\omega_c t + \phi \right)~.
\end{equation}
Since $ -\omega_c t = 2\pi T / \lambda$, the applied kick is 
\begin{equation}
  \Delta \left(\frac{\Delta E}{p_s c}\right) = \frac{V_0 \sin\left(\frac{2\pi \mathtt{T} }{\lambda}+\phi\right)}{p_0 c}~.
\end{equation}
Here it is assumed that $p_s = p_0$ and that the particle charge is $e$.
The design momentum is taken from the beam object, where it is calculated as described in Equation~\eqref{eq:designmomentum}.


TODO: Input

\subsection{RFcavity\_Matrix}

This element also implements an accelerating RF cavity, in a linearized approximation.
For this, a Taylor-expansion of the $\sin$ function is used, such that
\begin{equation}
  \begin{split}
    \sin\left(\frac{2\pi \mathtt{T} }{\lambda}+\phi\right) =& \sin\left(\frac{2\pi \mathtt{T} }{\lambda}\right)\cos\phi + \cos\left(\frac{2\pi \mathtt{T} }{\lambda}\right)\cos\phi \\
    =&
    \sum_{n=0}^\infty \left( \frac{(-1)^n}{(2n+1)!} \left(\frac{2\pi \mathtt{T}}{\lambda}\right)^{2n+1} \right) \cos\phi \\
    +& 
    \sum_{n=0}^\infty \left( \frac{(-1)^n}{(2n)!} \left(\frac{2\pi \mathtt{T}}{\lambda}\right)^{2n} \right) \sin\phi \\
    \approx& \cos\phi \left(\frac{2\pi \mathtt{T}}{\lambda}\right) + \sin\phi ~,
  \end{split}
\end{equation}
and the change in the last element of the vector is
\begin{equation}
  \Delta \left(\frac{\Delta E}{p_s c}\right) = \frac{V \cos\phi \left(\frac{2\pi \mathtt{T}}{\lambda}\right) + \sin\phi}{p_0 c} = \frac{V \cos\phi \left(\frac{2\pi }{\lambda}\right)}{p_0 c} \mathtt{T} + \frac{\sin\phi}{p_0c} \equiv \mathbb{R}_{6,5} \mathtt{T} + \mathbb{V}_6 ~.
\end{equation}
In general, this is an affine transform; however if $\phi = n\pi$ for some integer $n$ it reduces to a simple linear transform. Currently, the implementation assumes that $\phi = 0$.

This implementation returns a non-identity matrix when called by the \texttt{get\-Matrix}() method.

TODO: Input

\subsection{RFcavity\_loading}

This implements an accelerating RF cavity in the same way as described in Section~\ref{sec:elements:RFcavity}, however it also adds the effect of beam loading.
For a bunch that is much shorter than the RF wavelength, the amplitude of the beam induced voltage is (CITE the handbook, 2nd ed, section 2.4.3)
\begin{equation}
  V_{b0} = (R/Q) \omega_c q~,
  \label{eq:elements:RFcavityLoading:Vb0}
\end{equation}
where $\omega_c$ is the cavity resonant frequency, $q$ the bunch charge, and $R/Q$ the geometric factor of the cavity.
This voltage is deposited with a decelerating phase, such that a particle following immediatly after sees a voltage $-V_{b0}$.
Further, the decelerating voltage seen by each single bunch is $V_{b0}/2$, as shown by the fundamental theorem of beam loading.

Since the bunches we want to treat are not neccessarily short, Equation~\eqref{eq:elements:RFcavityLoading:Vb0} cannot be used directly.
It can however be used on a thin slice of the bunch, or for a single tracked particle.
The charge per tracked particle is then simply the beam charge divided by the number of particles in the bunch.

If the complex voltage (which is tracking both time and amplitude) at $t=0$ is $\tilde V_b(0)$, it hashas after a time $t$ developed as
\begin{equation}
  \tilde V_b(t) =  V_{b0} ~ e^{\mathbf{i}\pi} ~ \exp\left(-\mathrm{i}\omega_ct - \frac{t \omega_c}{2Q_L} \right)~.
  \label{eq:elements:RFcavityLoading:VbProp}
\end{equation}
Here, $Q_L = \omega_c U / P_\mathrm{loss}$ is the loaded Q-factor of the cavity, which gives the decay time $\tau_U = \frac{U}{\omega}$ of the energy $U$ contained in the cavitiy's field.
For a superconducting RF cavity, the radiation through the cavity coupler dominates the power loss, which therefore is the most important factor determining $Q_L$.
To get the actual voltage $V_b(t)$ that seen by the beam, it is neccessary to take the real part, i.e.\ $V_b(t) = \mathrm{Re}\left(\tilde V_b(t) \right)$.

Substituting $-\omega_c t = 2\pi \mathtt{T} / \lambda$, the voltage seen by each particle is thus given as
\begin{equation}
  V = V_g(\mathtt{T}) - V_{b0}/2 + \mathrm{Re}\left( \tilde V_b(\mathtt{T}=0) \exp\left(\frac{+\mathrm{i}~2\pi \mathtt{T}}{\lambda} + \frac{\pi \mathtt{T}}{\lambda Q_L} \right) \right)~.
\end{equation}
Here $\tilde V_b(\mathtt{T}=0)$ is the complex beam voltage at the center of the bunch, propagated from the previous bunch using Equation~\eqref{eq:elements:RFcavityLoading:VbProp} with $t=cL$ where $L$ is the distance between the bunches.
In the case of a single bunch, $L$ is the circumference of the machine.

After the treatment of each particle, the beam voltage $\tilde V_b$ is updated as
\begin{equation}
  \tilde V_b(T=0) - V_{b0} \exp \left( \frac{- \mathbf{i} ~ 2\pi \mathtt{T}}{\lambda}  \right)
\end{equation}
where the exponential propagates the induced voltage forward or backward in time to the ``centered'' time corresponding to $\mathtt{T}=0$.
Please note that the decay $Q_L$ is considered to be negliglible during the length of the bunch.

Due to causality, it is neccessary that the particles are treated in the order they arrive in the cavity, i.e.\ they must be reversed-sorted by \texttt{T}.
Furthermore, the value of $\omega_c$ should be carefully matched to the length of the machine and the harmonic number, otherwise the propagation of the voltage around one turn using \label{eq:elements:RFcavityLoading:VbProp} will give a phase offset every turn.

\subsection{CrabCavity}
\subsection{PrintMean}
\subsection{PrintBunch}

\chapter{Output and analysis tools}

\appendix
\chapter{Creating a machine definition from MadX}
\label{sec:MadX}

\chapter{Testing for correctness}
\section{Betatron tune}

\section{Synchrotron tune}

\end{document}