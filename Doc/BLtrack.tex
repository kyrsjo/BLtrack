\documentclass[a4paper]{report}
\title{BLtrack documentation}
\author{Kyrre Sjobak}

\usepackage{hyperref}
\usepackage{algorithmic}

\begin{document}
\maketitle
\newpage
\tableofcontents

\chapter{Introduction}

The BLtrack program is written to study the effect of beam loading in crab cavities.
It can track several bunches of particles in a 6D phase space, and allows the elements to change dynamically over time, including responding dynamically to the beam.
Most of the BLtrack input is meant to be exported from MadX, permitting easy set-up of a new configuration.

\section{Program structure}
The program structure is split in two main classes: \texttt{Beam} and \texttt{Ring}.
These then hold objects of classes inheriting from the \texttt{Bunch} and \texttt{Element} classes.

To start tracking, the \texttt{Machine} object has a method \texttt{track(beam,turns)} which tracks the provided \texttt{Beam} object through the ring the given number of turns.
The algorithm used to track all the particles through all the elements for the required number of turns is very simple:
\begin{algorithmic}
  \FOR{$t=0$ \TO $\mathtt{turns}-1$}
    \FOR{\texttt{element} in \texttt{elements}}
      \FOR{\texttt{bunch} in \texttt{bunches}}
        \STATE \texttt{bunch}.\texttt{particles} = \texttt{element}.\texttt{track}(\texttt{bunch},$t$)
      \ENDFOR
    \ENDFOR
  \ENDFOR
\end{algorithmic}

Finally there is the main executable, \texttt{BLtrack.py}.
This reads an input file, parses some sections of it directly, and passes the \texttt{BEAM} and \texttt{RING} blocks on to the relevant constructors.

\section{Input file structure}
The input file is based on newline-separated statements, identified by the first word in that statement.


\chapter{Beam definition}


\chapter{Machine elements}
\section{SectorMapMatrix}
\section{RFcavitiy}
\section{CrabCavity}
\section{PrintMean}
\section{PrintBunch}

\appendix
\chapter{Creating a machine definition from MadX}

\chapter{Testing for correctness}
\section{Betatron tune}

\section{Synchrotron tune}

\end{document}